\documentclass[10pt,a4paper]{article}
\usepackage[utf8]{inputenc}
\usepackage{amsmath}
\usepackage{amsfonts}
\usepackage{amssymb}
\author{Konrad Cybulski}
\title{Using agent-based simulations to verify when and if the mathematical
predictions of Santos, Santos and Pacheco both hold true and extend to a broader
range of parameters.}
\begin{document}
\begin{Large}
Using agent-based simulations to verify when and if the mathematical predictions of
Santos, Santos and Pacheco both hold true and extend to a broader range of parameters
\end{Large}
\section{Motivation}
In order to understand our behaviour, our willingness and ability to cooperate with those around us, it is vital that we can create simulations to model the dynamics of populations and the interactions of agents within them. It is also important to understand the way cooperation and defection evolve in both small and large populations. \\\\
These detailed and accurate simulations should be based on rigorous mathematical models which aim to explain and model how populations of agents using initially random strategies reach states with a high rate of cooperation. The area of indirect reciprocity using reputation in prisoner’s dilemma games played between individuals in a larger population has been the focus of recent research in game theory. \\\\
A number of papers have been published exploring the effect of strategies based on reputation in the prisoner’s dilemma game and a detailed look at those which foster cooperation. These papers include “\textit{The leading eight: Social norms that can maintain cooperation by indirect reciprocity}” by Hisashi Ohtsuki and Yoh Iwasa, “\textit{Evolution of indirect reciprocity}” by Martin A. Nowak and Karl Sigmund, “\textit{Evolution of cooperation through indirect reciprocity}” by Olof Leimar and Peter Hammerstein and numerous others. The main research paper on which this project will be based is “\textit{Social Norms of Cooperation in Small-Scale Societies}” by Santos, Santos, and Pacheco. \\\\
In order to create a platform for further research to be conducted on this model, it is necessary that the simulation itself is robust, fast, extensible and allows the user to test multiple conditions with ease. In addition to this, the results of the simulation should be evident and clear, and easily exportable to a number of data analysis packages. 
\section{Research Question}
What is the effect of reputation in fostering collaboration amongst decision makers? Verifying the results and simulations developed by Santos, Santos, Pacheco in their paper \textit{Social Norms of Cooperation in Small-Scale Societies}. 
\section{Aims}
The primary aim of this project is to replicate the findings of Santos, Santos, and Pacheco which will allow further exploration into more detailed models of the prisoner’s dilemma game within populations. The paper outlines simulation parameters based on equations to model the process in which the prisoner’s dilemma game is played over time in a population. In order to allow for greater research to be done in this area based on the simulation developed to verify the mathematical model developed by Santos et al. it is necessary that the simulation can be recreated and tested under similar variables and constraints to verify their results.
\section{Intended Outcomes}
The goal of this project is to write a program utilising a number of equations detailed in Santos et al. to model a prisoner’s dilemma game in a population which involved reputation dynamics. The simulation that will be produced should be efficient and extensible. The program itself will be programmed in Python and will utilise a number of libraries to maximise efficiency and speed of computation including \textit{numpy}$^{[3]}$ and \textit{Anaconda}$^{[4]}$. 
\section{Proposed Research Methodology}
The application to be developed in this project will be created in the Python programming language to allow for ease of prototype development. However, this program may be ported to a C++ base depending on the available speed of computation for the simulation to run.
There do exist a number of Python modules which allow a great speedup in computation time utilising CUDA cores on the GPU (such as Anaconda and Anaconda Accelerate) or simply compile Python code for faster computation (using modules such as Numba).
\section{Timeline}
\begin{tabular}{|l|c|c|c|c|c|c|c|c|c|c|c|c|}
\hline 
Week: & 1 & 2 & 3 & 4 & 5 & 6 & 7 & 8 & 9 & 10 & 11 & 12 \\ 
\hline 
Intitial meeting & - & - & • &  &  &  &  &  &  &  &  &  \\ 
\hline 
Review of literature & - & - & • & • &  &  &  &  &  &  &  &  \\ 
\hline 
Writing proposal & - & - & • & • &  &  &  &  &  &  &  &  \\ 
\hline 
Replication of simulation & - & - &  &  & • & • &  &  &  &  &  &  \\ 
\hline 
Optimisation of simulation & - & - &  &  &  &  & • & • &  &  &  &  \\ 
\hline 
Development of data analysis interface & - & - &  &  &  &  &  &  & • & • &  &  \\ 
\hline 
Project presentation & - & - &  &  &  &  &  &  &  &  & • &  \\ 
\hline 
Final report & - & - & • & • & • & • & • & • & • & • & • & • \\ 
\hline 
\end{tabular}

\pagebreak
\begin{thebibliography}{9}

\bibitem{1} 
Martin A. Novak, Karl Sigmund, (2005) 
\textit{Evolution of indirect reciprocity}. 
Nature.

\bibitem{2} 
Fernando P. Santos, Francisco C. Santos, Jorge M. Pacheco, (2016) 
\textit{Social Norms of Cooperation in Small-Scale Societies}. 
PLOS Computational Biology.

\bibitem{3} 
Numpy Developers, (2016) 
\textit{Numpy}. 
\\\texttt{http://www.numpy.org/}.

\bibitem{4} 
Continuum Analytics, Inc., (2016) 
\textit{Anaconda and Anaconda Accelerate}. 
\\\texttt{https://docs.continuum.io/}.

\bibitem{5} 
Hisashi Ohtsuki, Yoh Iwasa, (2005) 
\textit{The leading eight: Social norms that can maintain cooperation by indirect reciprocity}. 
Journal of Theoretical Biology.

\bibitem{6} 
Olof Leimar, Peter Hammerstein, (2001) 
\textit{Evolution of cooperation through indirect reciprocity}. 
Proceedings of the Royal Society B.

\end{thebibliography}
\end{document}